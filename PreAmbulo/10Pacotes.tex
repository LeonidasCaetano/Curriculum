%==================================================================
%pacotes
\usepackage[english,brazil]{babel}%Pacote BR
%=========================================================
\usepackage[normalem]{ulem}%sublinhar, etc
\usepackage{soul}
%=========================================================
\usepackage{mathtools}%equações matematicas
\usepackage{amsfonts}%letras matematicas
%=========================================================
%Tabelas
\usepackage{array}%basicão
%=========================================================
\usepackage[T1]{fontenc}%acentos
\usepackage{setspace}%definir distancia entre linhas, etc
\usepackage{anyfontsize}
\usepackage{indentfirst}%recuo 1º paragrafo
\usepackage{parskip}%distancia entre paragrafos
%=========================================================
%\usepackage{qrcode}%QRCode
%=========================================================
\usepackage{calc}%operação com variaveis
\usepackage[absolute]{textpos}%posicionar textos livremente
%=========================================================
\ifdefined\directlua
	\usepackage{fontspec}%mudar fonte
	\usepackage{pdflscape}
	\usepackage{luacode}
\fi
\ifdefined\pdfoutput
	\usepackage{helvet}
	\renewcommand{\familydefault}{\sfdefault}%Arial
	\usepackage{pdflscape}
\fi
\ifdefined\XeTeXversion
	\usepackage{mathspec}%mudar fonte
\fi
%\usepackage{contour}%fonte aberta
%=========================================================
\usepackage{pgffor}%loop option 1 
%\usepackage{forloop}%loop option 2
%==================================================================

\typeout{Pacotes carregados}